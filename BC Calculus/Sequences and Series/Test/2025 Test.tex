\documentclass[addpoints]{exam}
%%%%%%%%%%%%%%%%%% PACKAGES %%%%%%%%%%%%%%%%%%%%%%%%
\usepackage{amsmath}
\usepackage{amsfonts}
\usepackage{tcolorbox}
\usepackage{amssymb}
\usepackage{tikz,tkz-euclide,tikz-3dplot}
%%%%%%%%%%%%%%%%%%%%%% MARGINS%% %%%%%%%%%%%%%%%%%%%
\extrawidth{.5in}
\extraheadheight{-.25in}
\extrafootheight{-.25in}
%%%%%%%%%%%%%%%%%% ANSWERS AND POINTS %%%%%%%%%%%%%%
\printanswers
\pointsinrightmargin
\bracketedpoints
%%%%%%%%%%%%%%%%%% HEADER AND FOOTER %%%%%%%%%%%%%%%
\pagestyle{headandfoot}
\firstpageheadrule
\runningheadrule
\firstpageheader{\textbf{Calculator Allowed \& 65 Mins}\\Sequences and Series Test}{}{AP BC Calculus\\Mr. Carey}
\runningheader{Sequences and Series Test}{}{Mr. Carey}
\firstpagefooter{}{}{}
\runningfooter{ }{\thepage}{ }
%%%%%%%%%%%%%%%%%%%%%% NOTES %%%%%%%%%%%%%%%%%%%%%%%
%The following code is for formatting T/F Questions
\newcommand*{\TrueFalse}[1]{%
\ifprintanswers
    \ifthenelse{\equal{#1}{T}}{%
        \textbf{TRUE}\hspace*{14pt}False
    }{
        True\hspace*{14pt}\textbf{FALSE}
    }
\else
    {True}\hspace*{20pt}False
\fi
} 
%% The following code is based on an answer by Gonzalo Medina
%% https://tex.stackexchange.com/a/13106/39194
\newlength\TFlengthA
\newlength\TFlengthB
\settowidth\TFlengthA{\hspace*{1.16in}}
\newcommand\TFQuestion[2]{%
    \setlength\TFlengthB{\linewidth}
    \addtolength\TFlengthB{-\TFlengthA}
    \parbox[t]{\TFlengthA}{\TrueFalse{#1}}\parbox[t]{\TFlengthB}{#2}}
%%%%%%%%%%%%%%%%%% DOCUMENT CONTENTS %%%%%%%%%%%%%%%
\begin{document}
\vspace{1in}

\noindent\makebox[.75\textwidth]{Name:\enspace\hrulefill} \hspace{.5in} Grade:\enspace\hrulefill/\numpoints

\vspace{.1in}

\begin{center}
	\fbox{\fbox{\parbox{5.5in}{\centering
				Answer the questions in the spaces provided on the following pages.  If you run out of room for an answer, continue on the back of the page. Show \textbf{all} your work to be able to receive full credit on any question.
				\textbf{YOU ARE A MISSILE}}}}
\end{center}

\begin{questions}
	\fullwidth{For each statement below, circle whether they are true or false. As a reminder, a statement that is not always true is considered false. No work is needed for the following problems.}
	\question[3]\TFQuestion{T}{If $\displaystyle\lim_{n\to\infty}a_n=0$, then $\sum a_n$ is convergent.}
	\vspace{\stretch{1}}

	\question[3]\TFQuestion{T}{If $\sum a_n$ is divergent, then $\sum |a_n|$ is divergent.}
	\vspace{\stretch{1}}

	\question[3]\TFQuestion{T}{If $\sum a_n$ is divergent, then $\sum |a_n|$ is divergent.}
	\vspace{\stretch{1}}
	\question[3]\TFQuestion{T}{The ratio test can be used to determine whether $\sum 1/n!$ converges.}
	\vspace{\stretch{1}}
	\question[3]\TFQuestion{T}{The ratio test can be used to determine whether $\sum 1/n^3$ converges.}
	\vspace{\stretch{1}}
	\question[3]\TFQuestion{T}{If $0\le a_n\le b_n$ and $\sum b_n$ converges, then $\sum a_n$ converges.}
	\vspace{\stretch{1}}
	\question[3]\TFQuestion{T}{If $-1<\alpha<1$, then $\sum \alpha^n$ is convergent.}
	\vspace{\stretch{1}}
	\question[3]\TFQuestion{T}{If $a_n>0$ and $\displaystyle\lim_{n\to\infty}(a_{n+1}/a_n)=1$, then $\sum a_n$  converges.}
	\vspace{\stretch{1}}
	\question[3]\TFQuestion{T}{$\displaystyle\sum_{n=0}^{\infty}\frac{(-1)^n}{n!}=\frac{1}{e}$}
	\vspace{\stretch{1}}
	\question[3]\TFQuestion{T}{If a series converges, then the sequence of its terms converges.}
	\vspace{\stretch{1}}
	\question[3]\TFQuestion{T}{If a series converges, then the sequence of its partial sums converge.}
	\vspace{\stretch{1}}
	\question[3]\TFQuestion{T}{A series can be convergent without its sequence being convergent.}
	\vspace{\stretch{1}}
	\question[3]\TFQuestion{T}{If $a_n>0$ and $b_n>0$, $\sum b_n$ converges and $\displaystyle\lim_{n\to\infty}\frac{a_n}{b_n}=0$, then $\sum a_n$ is convergent.}
	\vspace{\stretch{1}}
	\question[3]\TFQuestion{T}{$\displaystyle\sin(x)=\sum_{n=0}^\infty\frac{(-1)^n x^{2n}}{(2n)!}$}
	\vspace{\stretch{1}}
	\question[3]\TFQuestion{T}{The alternating harmonic series is conditionally convergent.}
	\vspace{\stretch{1}}

	\newpage


	\question Determine whether the following series are convergent or divergent. Show all work to recieve full credit. An answer of convergent or divergent alone will yeild no credit.
	\begin{parts}
		\part[10] $\displaystyle\sum_{n=2}^\infty\frac{\ln n}{n^3}$

		\begin{solution}[\stretch{1}]
			\[
				\frac{\ln n}{n^{3}} < \frac{n}{n^3} \Rightarrow \sum_{n=2}^{\infty}\frac{\ln n}{n^{3}} \hspace{0.1in} \text{converges by DCT since } \sum_{n=2}^{\infty} \frac{n}{n^3} \hspace{0.1in} \text{converges by $p$-series}
			\]
		\end{solution}

		\part[10]$\displaystyle\sum_{i=1}^\infty \left(\sqrt[i]{i}-1\right)^i$

    \begin{solution}[\stretch{1}]
      \[
        L = \lim_{i\to\,\infty} \sqrt[i]{\left|\left(\sqrt[i]{i} - 1\right)^i\right|} = \lim_{i\to\,\infty} \left|\sqrt[i]{i} - 1\right| = 0
      \]
      $\therefore L < 1$ implies that the series is convergent by the root test
    \end{solution}

		\part[10]$\displaystyle\sum_{n=0}^\infty (-1)^{n+1} e^{\left(\frac{2}{n+1}\right)}$

    \begin{solution}[\stretch{1}]
      \[
        \lim_{n\to\,\infty} e^{\left(\frac{2}{n+1}\right)} = 1 \ne 0 
      \]
      $\therefore$ since the limit of the non alternating term is not, $0$, the series diverges by the AST
    \end{solution}

		\part[10]$\displaystyle\sum_{k=1}^\infty \frac{\arctan(k)}{k^{1.5}}$

    \begin{solution}[\stretch{1}]
      Let $b_n = \frac{1}{k^{1.5}}$
      \[
        \lim_{k\to\infty} \frac{a_k}{b_k} = \lim_{k\to\,\infty} = \frac{\arctan(k)}{k^{1.5}} \frac{k^{1.5}} = \frac{\pi}{2}
      \]
      $\therefore$ the series converges by LCT
    \end{solution}
	\end{parts}

	\newpage

	\question[10] Find the sum of the following convergent series.
	\begin{parts}
		\begin{minipage}[t]{.45\linewidth}
			\part $\displaystyle\sum_{n=0}^\infty \frac{3^{n+1}}{(-4)^n}$
		\end{minipage}
		\hfill
		\begin{minipage}[t]{.45\linewidth}
			\part $\displaystyle\sum_{i=2}^\infty \left[\frac{1}{i^2}-\frac{1}{(i+2)^2}\right]$
		\end{minipage}
	\end{parts}

	\vspace{\stretch{1}}

	\question[5] Find the raidus of convergence for the given power series.\\ $\displaystyle\sum_{n=0}^\infty \frac{2^n}{n!}(x-1)^n$

	\vspace{\stretch{.5}}

	\question[10] Find the interval of convergence for the given power series.\\ $\displaystyle\sum_{n=2}^\infty \frac{n}{n+1}\left(\frac{x}{4}\right)^{n-1}$

	\vspace{\stretch{1}}

	\newpage

	\question[15] Consider the function given by
	\[f(x)=\sum_{n=0}^\infty\frac{(-1)^n (x+2)^{n+1}}{n+1}.\]
	Answer the following.
	\begin{parts}
		\part Find a series for $f'(x)$. Then, find its interval of convergence.

		\vspace{\stretch{1}}

		\part Write the series $\displaystyle\int_{-2}^{-1.3}f(x)dx$. You do not need to find the interval of convergence.

		\vspace{\stretch{1}}

		\part Show that the maximum error associated with the approximation of $S_7$ for the series obtained in part (b) is less than $\displaystyle\frac{29}{500}.$

		\vspace{\stretch{1}}

	\end{parts}

	\newpage

	\question[30] Consider the following functions. Using any means, construct a power series for the given function at the specified center. Leave your answer in summation notation. Simplify completely.

	\begin{parts}
		\part $\displaystyle h(x)=3xe^{-\frac{x^2}{3}},\,c=0$

		\vspace{\stretch{1}}

		\part $\displaystyle r(x)=\frac{1}{1+x},\,c=3$

		\vspace{\stretch{1}}

		\part $\displaystyle t(x)=\sin3x\cos3x,\,c=0$

		\vspace{\stretch{1}}
	\end{parts}

	\newpage

	\question Let $f$ be the function given by $\displaystyle f(x)=\frac{1}{1-2x}$ and let $P_3(x)$ be the third degree MacLaurin Polynomial for $f$.
	\begin{parts}
		\part[10] Find $P_3(x)$.

		\vspace{\stretch{1}}

		\part[15] Use the Lagrange Error Bound to show that $\displaystyle\left|f\left(\frac{1}{6}\right)-P_3\left(\frac{1}{6}\right)\right|\le\frac{1}{1800}$.

		\vspace{\stretch{1}}
	\end{parts}

	\newpage

	\question The function $\phi$ has a Taylor series about $x=2$ that converges for all $x$ in it's interval of convergence. The $n$th derivative of $\phi$ at $x=2$ is given by \[\phi^{(n)}(2)=\frac{(n+1)!}{3^n}\] for $n\ge1$.
	\begin{parts}
		\part[10] Find an expression for the Lagrange Error bound associated with the approximation of $\phi(3)$ using the $n$th degree Taylor polynomial centered about $x=2$.

		\vspace{\stretch{1}}

		\part[10] How many terms are needed for the error to be less than $\displaystyle\frac{1}{1200}$?

		\vspace{\stretch{.5}}
	\end{parts}


\end{questions}

\newpage

\noindent\textbf{Extra Credit (10pts):} Consider the following series. \[\sum_{n=0}^\infty\frac{n!\cdot(n+1)!\cdot3^n}{\left(1\cdot3\cdot6\cdot\cdots\cdot(3n)\right)^2}.\] Determine the convergence or divergence of the series. Show all work.
\vspace{\stretch{1}}








\end{document}
