\documentclass[addpoints]{exam}
%%%%%%%%%%%%%%%%%% PACKAGES %%%%%%%%%%%%%%%%%%%%%%%%
\usepackage{amsmath}
\usepackage{amsfonts}
\usepackage{tcolorbox}
\tcbuselibrary{skins}
\usepackage{tikz,tkz-euclide,tikz-3dplot}
\usepackage{rotating}
%%%%%%%%%%%%%%%%%%%%%% MARGINS%% %%%%%%%%%%%%%%%%%%%
\extrawidth{.5in}
\extraheadheight{-.25in}
\extrafootheight{-.25in}
%%%%%%%%%%%%%%%%%% ANSWERS AND POINTS %%%%%%%%%%%%%%
%\printanswers
\pointsinrightmargin
\bracketedpoints
%%%%%%%%%%%%%%%%%% HEADER AND FOOTER %%%%%%%%%%%%%%%
\pagestyle{headandfoot}
\firstpageheadrule
\runningheadrule
\firstpageheader{\S4.A.3 Other Series}{}{AP Precalc\\Mr. Carey}
\runningheader{\S4.A.3 Other Series}{}{Mr. Carey}
\firstpagefooter{}{}{}
\runningfooter{ }{\thepage}{ }
%%%%%%%%%%%%%%%%%%%%%% NOTES %%%%%%%%%%%%%%%%%%%%%%%

%%%%%%%%%%%%%%%%%% DOCUMENT CONTENTS %%%%%%%%%%%%%%%
\begin{document}
\section*{Other Common Sequences and Series}
\subsection*{Factorials}
While this particular class does not focus on them much, they become important in calculus and probability.

\begin{tcolorbox}[title=Definition: \textit{Factorial},title filled,colframe=black,sharpish corners,width=\linewidth]
    For a nonnegative integer $n$, the \textbf{factorial} of $n$ is defined as
    \[n! = n(n-1)(n-2)\cdots(3)(2)(1)\]
    with the special case that $0! = 1$.\\
    \\
    The factorial of $n$ is the product of all positive integers less than or equal to $n$, which is a recursive sequence: $n! = n(n-1)!$.
\end{tcolorbox}
\noindent Simplify the following factorials.
\begin{questions}
    \begin{minipage}{.3\linewidth}
        \question $\displaystyle\frac{8!}{2!\cdot6!}$
    \end{minipage}
    \hfill
    \begin{minipage}{.3\linewidth}
        \question $\displaystyle\frac{n!}{(n-1)!}$
    \end{minipage}
    \hfill
    \begin{minipage}{.3\linewidth}
        \question $\displaystyle\frac{4!(n+2)!}{6!n!}$
    \end{minipage}

    \vspace{\stretch{1}}
\end{questions}

\subsection*{Binomial Coefficient}
\begin{tcolorbox}[title=Definition: \textit{Combinations},title filled,colframe=black,sharpish corners,width=\linewidth]
    For a nonnegative integers $n$ and $r$, with $0\le r\le n$, then \[_nC_r=\frac{n!}{(n-r)!r!}.\] The symbol $\displaystyle\binom{n}{r}$ is often used of the $C$ notation when referring to the binomial coefficient. 
\end{tcolorbox}
\noindent Evaluate each of the following expressions.
\begin{questions}
    \begin{minipage}{.2\linewidth}
        \question $\displaystyle_8C_2$
    \end{minipage}
    \hfill
    \begin{minipage}{.2\linewidth}
        \question $\displaystyle\binom{10}{3}$
    \end{minipage}
    \hfill
    \begin{minipage}{.2\linewidth}
        \question $\displaystyle_7C_4$
    \end{minipage}
    \begin{minipage}{.2\linewidth}
        \question $\displaystyle_7C_3$
    \end{minipage}

    \vspace{\stretch{1}}
\end{questions}

\newpage

\subsection*{The Binomial Theorem}
\noindent\textbf{Introductory Example:}\\ 
By repeated foiling, expand the expression $(x+y)^3$. Then, compute the values of $\displaystyle\binom{3}{0},\,\binom{3}{1},\,\binom{3}{2},$ and $\displaystyle\binom{3}{3}$. What do you notice?

\vspace{\stretch{1}}

\begin{tcolorbox}[title=Definition: \textit{The Binomial Theorem},title filled,colframe=black,sharpish corners,width=\linewidth]
For all $n\in\mathbb{Z}^+$, the expansion of $(x+y)^n$ is \[(x+y)^n=x^n+nx^{n-1}y+\cdots+\binom{n}{r}x^{n-r}y^r+\cdots+nxy^{n-1}+y^n.\] The Binomial Theorem allows us to expand, or partially expand, binomials quickly or to higher degrees without foiling repeatedly.
\end{tcolorbox}

\noindent\textbf{Practice:} .
\begin{questions}
    \question Expand each expression
    
    \begin{parts}
        \part $\displaystyle(2x+1)^4$

        \vspace{\stretch{.6}}

        \part $\displaystyle(3x-2y)^5$

        \vspace{\stretch{.6}}

        \part $\displaystyle\left(x^{2}+\frac{1}{x}\right)^6$

        \vspace{\stretch{.6}}
    \end{parts}



\newpage

    \question Find the sixth term of the expansion of $(a+2b)^8$.

    \vspace{\stretch{1}}

    \question Find the coefficient of the $a^6b^5$ term in the expansion of $(2a-5b)^11$.

    \vspace{\stretch{1}}

    \question Find the coefficient of the constant term in the expansion of $\left(3x - \frac{2}{x^2}\right)^{10}$.

    \vspace{\stretch{1}}

\end{questions}


\end{document}