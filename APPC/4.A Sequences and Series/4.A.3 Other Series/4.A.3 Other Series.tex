\documentclass[addpoints]{exam}
%%%%%%%%%%%%%%%%%% PACKAGES %%%%%%%%%%%%%%%%%%%%%%%%
\usepackage{amsmath}
\usepackage{amsfonts}
\usepackage{tcolorbox}
\tcbuselibrary{skins}
\usepackage{tikz,tkz-euclide,tikz-3dplot}
\usepackage{rotating}
%%%%%%%%%%%%%%%%%%%%%% MARGINS%% %%%%%%%%%%%%%%%%%%%
\extrawidth{.5in}
\extraheadheight{-.25in}
\extrafootheight{-.25in}
%%%%%%%%%%%%%%%%%% ANSWERS AND POINTS %%%%%%%%%%%%%%
%\printanswers
\pointsinrightmargin
\bracketedpoints
%%%%%%%%%%%%%%%%%% HEADER AND FOOTER %%%%%%%%%%%%%%%
\pagestyle{headandfoot}
\firstpageheadrule
\runningheadrule
\firstpageheader{\S4.A.3 Other Series}{}{AP Precalc\\Mr. Carey}
\runningheader{\S4.A.3 Other Series}{}{Mr. Carey}
\firstpagefooter{}{}{}
\runningfooter{ }{\thepage}{ }
%%%%%%%%%%%%%%%%%%%%%% NOTES %%%%%%%%%%%%%%%%%%%%%%%

%%%%%%%%%%%%%%%%%% DOCUMENT CONTENTS %%%%%%%%%%%%%%%
\begin{document}
\section*{Other Common Sequences and Series}
\subsection*{Factorials}
While this particular class does not focus on them much, they become important in calculus and probability.

\begin{tcolorbox}[title=Definition: \textit{Factorial},title filled,colframe=black,sharpish corners,width=\linewidth]
    For a nonnegative integer $n$, the \textbf{factorial} of $n$ is defined as
    \[n! = n(n-1)(n-2)\cdots(3)(2)(1)\]
    with the special case that $0! = 1$.\\
    \\
    The factorial of $n$ is the product of all positive integers less than or equal to $n$, which is a recursive sequence: $n! = n(n-1)!$.
\end{tcolorbox}
\noindent Simplify the following factorials.
\begin{questions}
    \begin{minipage}{.3\linewidth}
        \question $\displaystyle\frac{8!}{2!\cdot6!}$
    \end{minipage}
    \hfill
    \begin{minipage}{.3\linewidth}
        \question $\displaystyle\frac{n!}{(n-1)!}$
    \end{minipage}
    \hfill
    \begin{minipage}{.3\linewidth}
        \question $\displaystyle\frac{4!(n+2)!}{6!n!}$
    \end{minipage}

    \vspace{\stretch{1}}
\end{questions}






\end{document}