\documentclass{exam}
%%%%%%%%%%%%%%%%%% PACKAGES %%%%%%%%%%%%%%%%%%%%%%%%
\usepackage{amsmath}
\usepackage{amsfonts}
\usepackage{tcolorbox}
\usepackage{tikz,tkz-euclide,tikz-3dplot}
%%%%%%%%%%%%%%%%%%%%%% MARGINS%% %%%%%%%%%%%%%%%%%%%
\extrawidth{.5in}
\extraheadheight{-.25in}
\extrafootheight{-.25in}
%%%%%%%%%%%%%%%%%% ANSWERS AND POINTS %%%%%%%%%%%%%%
\printanswers
%%%%%%%%%%%%%%%%%% HEADER AND FOOTER %%%%%%%%%%%%%%%
\pagestyle{headandfoot}
\firstpageheadrule
\runningheadrule
\firstpageheader{\S3.EF IDH Key}{}{AP Precalc\\Mr. Carey}
\runningheader{\S3.EF IDH Key}{}{Mr. Carey}
\firstpagefooter{}{}{}
\runningfooter{ }{\thepage}{ }
%%%%%%%%%%%%%%%%%% DOCUMENT CONTENTS %%%%%%%%%%%%%%%
\begin{document}
\noindent For any of the verifying identities, the work can be done differently but still arive at the same answer. The following are just one way to verify the identity. The work below is meant to be a guide, not a definitive way to get the answer. Many approaches work.\\
\begin{questions}
\question $\displaystyle\frac{1-\tan^2 x}{1+\tan^2 x}=\cos^2 x-\sin^2 x$
\begin{solution}
\begin{align*}
\frac{1-\tan^2 x}{1+\tan^2 x}&=\frac{1-\tan^2 x}{\sec^2 x}\\
&=\frac{1}{\sec^2 x}-\frac{\tan^2 x}{\sec^2 x} \\
&=\cos^2 x-\frac{\sin^2 x}{\cos^2 x}\cdot\cos^2 x\\
&=\cos^2 x-\sin^2 x
\end{align*}
\end{solution}

\question $\displaystyle\frac{1+\tan x+\sec x}{1+\tan x+\sec x}=(1+\sec x)(1-\csc x)$
\begin{solution}
\begin{align*}
\frac{1+\tan x+\sec x}{1+\tan x-\sec x}&=\frac{1+\tan x+\sec x}{1+\tan x-\sec x}\cdot\frac{\cos x}{\cos x}\\
&=\frac{\cos x+\sin x +1}{\cos x +\sin x-1}\\
&=\frac{1+\sin x +\cos x}{\cos x +\sin x-1}\cdot\frac{\cos x+\sin x +1}{\cos x+\sin x + 1}\\
&=\frac{(\cos x+\sin x +1)^2}{(\cos x +\sin x)^2-1}\\
&=\frac{\cos^2 x+2\cos x\sin x+\sin^2 x+2\cos x+2\sin x+1}{\cos^2 x+2\cos x\sin x+\sin^2 x-1}\\
&=\frac{1+2\cos x\sin x+2\cos x+2\sin x+1}{1 +2\cos x\sin x-1}\\
&=\frac{2\cos x\sin x+2\cos x+2\sin x+2}{2\cos x\sin x}\\
&=\frac{\cos x\sin x+\cos x+\sin x+1}{\cos x\sin x}\\
&=\frac{\cos x(\sin x+1)+\sin x+1}{\cos x\sin x}\\
&=\frac{(\cos x+1)(\sin x+1)}{\cos x\sin x}\\
&=\left(\frac{\cos x+1}{\cos x}\right)\left(\frac{\sin x+1}{\sin x}\right)\\
&=(1+\sec x)(1+\csc x)
\end{align*}
\end{solution}

\newpage

\question $\displaystyle 7\sec^2 x-6\tan^2x+9\cos^2x=\frac{\left(1+3\cos^2x\right)^2}{\cos^2x}$
\begin{solution}
\begin{align*}
7\sec^2 x-6\tan^2x+9\cos^2x&=7\sec^2 x-6(\sec^2 x-1)+9\cos^2x\\
&=7\sec^2 x-6\sec^2 x+6+9\cos^2x\\
&=\sec^2 x+6+9\cos^2x\\
&=\frac{1}{\cos^2 x}+6+9\cos^2x\\
&=\frac{1+6\cos^2 x+9\cos^4 x}{\cos^2 x}\\
&=\frac{(1+3\cos^2 x)^2}{\cos^2 x}
\end{align*}
\end{solution}

\question $\displaystyle\tan(\alpha+\beta)=\frac{\tan\alpha+\tan\beta}{1-\tan\alpha\tan\beta}$
\begin{solution}
\begin{align*}
\tan(\alpha+\beta)&=\frac{\sin(\alpha+\beta)}{\cos(\alpha+\beta)}\\
&=\frac{\sin\alpha\cos\beta+\cos\alpha\sin\beta}{\cos\alpha\cos\beta-\sin\alpha\sin\beta}\\
&=\frac{\sin\alpha\cos\beta+\cos\alpha\sin\beta}{\cos\alpha\cos\beta-\sin\alpha\sin\beta}\cdot\frac{\frac{1}{\cos\alpha\cos\beta}}{\frac{1}{\cos\alpha\cos\beta}}\\
&=\frac{\frac{\sin\alpha}{\cos\alpha}+\frac{\sin\beta}{\cos\beta}}{1-\frac{\sin\alpha\sin\beta}{\cos\alpha\cos\beta}}\\
&=\frac{\tan\alpha+\tan\beta}{1-\tan\alpha\tan\beta}
\end{align*}
\end{solution}

\question $\displaystyle\tan\left(\frac{7\pi}{12}\right)$
\begin{solution}
\begin{align*}
\tan\left(\frac{7\pi}{12}\right)&=\tan\left(\frac{3\pi}{4}+\frac{\pi}{6}\right)\\
&=\frac{\tan\frac{3\pi}{4}+\tan\frac{\pi}{6}}{1-\tan\frac{3\pi}{4}\tan\frac{\pi}{6}}\\
&=\frac{-1+\frac{1}{\sqrt{3}}}{1+\frac{1}{\sqrt{3}}}\cdot\frac{\sqrt{3}}{\sqrt{3}}\\
&=\frac{1-\sqrt{3}}{1+\sqrt{3}}\cdot\frac{1-\sqrt{3}}{1-\sqrt{3}}\\
&=-2-\sqrt{3}
\end{align*}
\end{solution}

\newpage

\question Given that $\cos\alpha=-0.1$ and $\sin\beta=0.2$, find the exact value of $\cos(\alpha+\beta)$ if $\pi<\alpha<\frac{3\pi}{2}$ and $0<\beta<\frac{\pi}{2}$.
\begin{solution}
Since $\cos\alpha=-0.1=-\frac{1}{10}$ and $\sin\beta=0.2=\frac{1}{5}$, we can find $\sin\alpha$ and $\cos\beta$. Since $\alpha$ is in the third quadrant and $\beta$ is in the first quadrant, $\sin\alpha$ is negative and $\cos\beta$ is positive.
\begin{align*}
\sin\alpha&=-\sqrt{1-\cos^2\alpha}=-\sqrt{1-\left(-\frac{1}{10}\right)^2}=-\sqrt{1-\frac{1}{100}}=-\sqrt{\frac{99}{100}}=-\frac{3\sqrt{11}}{10}\\
\cos\beta&=\sqrt{1-\sin^2\beta}=\sqrt{1-\left(\frac{1}{5}\right)^2}=\sqrt{1-\frac{1}{25}}=\sqrt{\frac{24}{25}}=\frac{2\sqrt{6}}{5}
\end{align*}

Since $\alpha$ is in the third quadrant and $\beta$ is in the first quadrant, $\alpha+\beta$ is in the fourth quadrant. So our answer should be positive.

Now we can find $\cos(\alpha+\beta)$.
\begin{align*}
\cos(\alpha+\beta)&=\cos\alpha\cos\beta-\sin\alpha\sin\beta\\
&=-\frac{1}{10}\cdot\frac{2\sqrt{6}}{5}-\left(-\frac{3\sqrt{11}}{10}\right)\cdot\frac{1}{5}\\
&=-\frac{2\sqrt{6}}{50}+\frac{3\sqrt{11}}{50}\\
&=\frac{-2\sqrt{6}+3\sqrt{11}}{50}
\end{align*} 
\end{solution}

\question Given that $\sec u=-3$ and $\pi<u<\frac{3\pi}{2}$, find the exact value of $\sin2u,\,\cos2u,\text{ and }\tan2u$.
\begin{solution}
Since $\sec u=-3$, we can find $\cos u$ and $\sin u$. Since $\sec u=-3$ is negative, $\cos u$ is negative. Since $u$ is in the third quadrant, $\sin u$ is negative.
\begin{align*}
\cos u&=\frac{1}{\sec u}=\frac{1}{-3}=-\frac{1}{3}\\
\sin u&=-\sqrt{1-\cos^2 u}=-\sqrt{1-\left(-\frac{1}{3}\right)^2}=-\sqrt{1-\frac{1}{9}}=-\sqrt{\frac{8}{9}}=-\frac{2\sqrt{2}}{3}
\end{align*}
Now we can find $\sin2u,\,\cos2u,\text{ and }\tan2u$.

\begin{minipage}[t]{.3\linewidth}
    \begin{align*}
    \sin2u&=2\sin u\cos u\\
    &=2\left(-\frac{2\sqrt{2}}{3}\right)\left(-\frac{1}{3}\right)\\
    &=\frac{4\sqrt{2}}{9}\\
    \end{align*}
\end{minipage}
\hfill
\begin{minipage}[t]{.3\linewidth}
    \begin{align*}
    \cos2u&=\cos^2 u-\sin^2 u\\
    &=\left(-\frac{1}{3}\right)^2-\left(-\frac{2\sqrt{2}}{3}\right)^2\\
    &=\frac{1}{9}-\frac{8}{9}\\
    &=-\frac{7}{9}
    \end{align*}
\end{minipage}
\hfill
\begin{minipage}[t]{.3\linewidth}
    \begin{align*}
    \tan2u&=\frac{\sin2u}{\cos2u}\\
    &=\frac{\frac{4\sqrt{2}}{9}}{-\frac{7}{9}}\\
    &=-\frac{4\sqrt{2}}{7}
    \end{align*}
\end{minipage}
\end{solution}

\newpage

\question Find all solutions to the equation $2\cos x+\sin 2x=0$.
\begin{solution}
\begin{align*}
2\cos x+\sin 2x&=0\\
2\cos x+2\sin x\cos x&=0\\
2\cos x(1+\sin x)&=0
\end{align*}
So either $\cos x=0$ or $1+\sin x=0$. If $\cos x=0$, then $x=\frac{\pi}{2}+n\pi$ for any integer $n$. If $1+\sin x=0$, then $\sin x=-1$ so $x=\frac{3\pi}{2}+2n\pi$ for any integer $n$. So the solutions are $x=\frac{\pi}{2}+n\pi$ and $x=\frac{3\pi}{2}+2n\pi$ for any integer $n$.
\end{solution}

\question Find all solutions to $\displaystyle\tan^2\left(\frac{\pi}{8}\left(x-3\right)\right)=1$ on the interval $0\le x<8$.
\begin{solution}
\begin{align*}
\tan^2\left(\frac{\pi}{8}\left(x-3\right)\right)&=1\\
\tan\left(\frac{\pi}{8}\left(x-3\right)\right)&=\pm1\\
\frac{\pi}{8}\left(x-3\right)&=\frac{\pi}{4}+\pi n\text{ or }\frac{\pi}{8}\left(x-3\right)=-\frac{\pi}{4}+\pi n\text{ for any integer }n\\
x-3&=2+8n\text{ or }x-3=-2+8n\\
x&=5+8n\text{ or }x=1+8n
\end{align*}
Since $0\le x<8$, the solutions are $x=1$ and $x=5$.

\end{solution}


\end{questions}
\end{document}